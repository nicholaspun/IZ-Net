\section{Triplet Mining (with painfully too much tech)}

In part of training using triplet loss, we must choose appropriate triplets to feed into our network.
There are countless guides and explanations online for how to produce the so-called \it{semi-hard} and \it{hard} triplets (over the \it{easy} ones).
But, such is not the focus of this section, instead we are interested in how to choose a uniformly random triplet.
Clearly, producing every possible triplet and picking from that set is not very kind to a computer's memory, so the goal is to come up with a scheme that allows to avoid this.

Suppose we have $m$ classes $C_1, \ldots, C_m$, not all necessarily the same size.
For convenience, define $\mathscr{C}_i = \bigcup_{\substack{k \neq i \\ k \in \{1, \ldots m\}}} C_k$.
Consider the following selection scheme:

\begin{algorithm}
    \caption{Uniform Triplet Mining (Take 1)}\label{Algorithm:appendix-triplet-mining:uniform-triplet-mining-take-one}

    \nl Select $i, j \longleftarrow {1, \ldots m}$ uniformly at random (just select one and then the other here). $i$ is the index of the class that will become the anchor

    \nl Select an random pair of objects in $C_i$ and a random object from $C_j$
\end{algorithm}

Seems simple enough and also not actually uniformly random.
This becomes apparent with a small example: 

\begin{changemargin}{1cm}{1cm}
Say $C_1 = \{\Gamma, \Lambda\}$ and $C_2 = \{\star, \Diamond, \Box\}$.
There are a total of $9$ different triplets and more importantly the pair $(\Gamma, \Lambda)$ participates in $3$ of them.
However, using \Cref{Algorithm:appendix-triplet-mining:uniform-triplet-mining-take-one} above, we have a $100\%$ chance of selecting $(\Gamma, \Lambda)$ once we choose $C_1$ as the anchor, which happen $50\%$ of the time.
As such, the triplets involving the pair $(\Gamma, \Lambda)$ are represented $50\%$ of the time under this scheme, which is not the $\frac{3}{9} \approx 33.33\%$ we calculated above!
\end{changemargin}

There are two issues with \Cref{Algorithm:appendix-triplet-mining:uniform-triplet-mining-take-one} above:
\begin{enumerate}
    \item The selection of $j$ is unnecessary: once a pair has been chosen from $C_i$, the remaining elements in $\mathscr{C}_i$ are all uniformly represented.
    
    So, we only need to choose $i$, select a pair from $C_i$ and then select an object uniformly at random from $\mathscr{C}_i$

    \item We cannot assign a uniform distribution to the indices $\{1, \ldots, m\}$.
    Referring back to our example above, note that $C_2$ produces $3$ distinct pairs while $C_1$ only produces $1$.
    Intuitively, this means we should select $C_2$ as the anchor more often (in particular, three times as often) so that each pair within the class is equally represented.

    Indeed, let $\mathcal{C} = \binom{C_1}{2} + \ldots \binom{C_m}{2}$.
    Then, set the probability of choosing $C_i$ as the anchor class to be $\frac{\binom{C_i}{2}}{\mathcal{C}}$.
\end{enumerate}

These two modifications give us the following:

\begin{algorithm}
    \caption{Uniform Triplet Mining (Take 2)}\label{Algorithm:appendix-triplet-mining:uniform-triplet-mining-take-two}

    \nl Select $C_i$ as the anchor class with probability $\frac{\binom{C_i}{2}}{\mathcal{C}}$

    \nl Select (uniformly at random) a pair of objects from $C_i$ and (again, uniformly at random) an object from $\mathscr{C}_i$
\end{algorithm}

It remains to figure out how to actually perform the first step of \Cref{Algorithm:appendix-triplet-mining:uniform-triplet-mining-take-two}.
To do this, we use a data structure known as a Segment Tree \footnote{See \href{https://github.com/nicholaspun/IZ-Net/blob/master/faceRecognition/SegmentTree.py}{\inlinecode{SegmentTree.py}}}.
We'll leave the details of the data structure for the reader to explore, but in summary:
\begin{itemize}
    \item A Segment Tree \tt{T} is created from a set of intervals (over $\R$)
    \item Given a point $\tt{v} \in \R$, the operation \tt{Find(T, v)} returns a list of intervals from \tt{T} containing \tt{v}
\end{itemize}

We will find that our particular implementation of the Segment Tree is very specialized: we require the intervals to be disjoint and to complete the real line.

With the preparatory work out of the way, we now describe our strategy for choosing the anchor class:
\begin{enumerate}
    \item Compute the probabilities for each $C_i$. For convenience, we'll denote these with $p_i$ so we don't have to repeatedly write out the fraction.

    \item Give each $C_i$ an interval from $[0, 1)$. This is easy: give $C_1$ the interval $[0, p_1)$, give $C_2$ the interval $[p_1, p_1 + p_2)$, and so on.
    
    Note that $\sum_{i = 1}^m p_i = 1$, so $C_m$ is given the interval $[p_1 + \ldots + p_{m-1}, 1)$, completing the interval $[0, 1)$

    \item Create a Segment Tree \tt{T} from these intervals.
    Note that we cheat a bit here and give $C_1$ the interval $(-\infty, p_1)$ and $C_m$ the interval $[p_1 + \ldots + p_{m-1}, \infty)$.
    This does not affect the probabilities, it is only an consequence of implementation.
\end{enumerate}

Now, pick a random number between $\tt{v} \in [0,1]$ (say, using the \href{https://docs.python.org/3/library/random.html}{\inlinecode{random}} module in Python).
The operation \tt{Find(T, v)} will return exactly one interval, and each interval corresponds to exactly one class.
As such, \tt{Find(T, v)} selects classes according to the probability distribution we've defined!

This completes \Cref{Algorithm:appendix-triplet-mining:uniform-triplet-mining-take-two}.
We end the section with two remarks:
\begin{enumerate}
    \item As the section title suggests, is painfully too much tech to choose triplets.
    For the most part, with large enough datasets and close-to-equal distribution of objects between the classes, these steps can be safely ignored.
    Not using such tooling will skew the data distribution a tad bit, but the effect on the probabilities will be by very \it{very} small fractional amounts.

    \item We employ a similar strategy for choosing pairs for binary classification, which can be found in \href{https://github.com/nicholaspun/IZ-Net/blob/master/faceRecognition/PairPickerHelper.py}{\inlinecode{PairPickerHelper.py}}.
    The code for triplet picking can be found in \href{https://github.com/nicholaspun/IZ-Net/blob/master/faceRecognition/TripletPickerHelper.py}{\inlinecode{TripletPickerHelper.py}}
\end{enumerate}
